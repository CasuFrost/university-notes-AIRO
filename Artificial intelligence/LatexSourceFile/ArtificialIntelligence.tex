\documentclass[10pt, letterpaper]{report}
% !TeX program = xelatex
%==================PREAMBOLO=======================%
\input{../../preamble/preamble.tex}
\newcommand{\titolo}{Artificial intelligence }

 %TOGLI COMMENTO SE USI XELATEX
%\usepackage{fontspec}
\title{\titolo} %========TITOLO========%
\author{Marco Casu}
\date{\vspace{-5ex}}
\begin{document}

%==================COPERTINA=======================%
\begin{titlepage}
    
\begin{center}
    %TOGLI COMMENTO SE USI XELATEX
   %\setmainfont{Palace Script MT}
   \HUGE Marco Casu\acc
\end{center}
\thispagestyle{empty}
\begin{figure}[h]
    \centering{
        %l'immagine deve avere una risoluzione 2048x2048
        \includegraphics[width=1\textwidth ]{images/Copertina.png}
    }
\end{figure}
\vfill 
\centering \includegraphics[width=0.4\textwidth ]{../../preamble/Stemma_sapienza.png} \acc
\centering \Large \color{sapienza}Faculty of Information Engineering, Computer Science and Statistics\\
Department of Computer, Control and Management Engineering\\
Master's degree in Artificial Intelligence and Robotics
\end{titlepage}

%===================FINE COPERTINA======================%
\newpage
%\pagecolor{cartaRiciclata}%\setmainfont{Algerian}
\Large
This document summarizes and presents the topics for the \titolo course for the Master's degree in Artificial Intelligence and Robotics at Sapienza University of Rome. The document is free for any use. If the reader notices any typos, they are kindly requested to report them to the author.
\vfill
\begin{figure}[h!]
    \raggedright
    \includegraphics[width=0.4\textwidth,right ]{../../preamble/tomodachi.pdf} 
\end{figure}
\newpage %\setmainfont{Times New Roman}
\normalsize

\tableofcontents 
\newpage

%==================FOOTER e HEADER=======================%
\fancyhf{}
\fancyhead[L]{\nouppercase{\leftmark}}
\fancyhead[R]{Sezione \thesection}
\fancyfoot[C]{\thepage}
\fancyfoot[L]{\titolo}
\fancyfoot[R]{ Marco Casu}
%\fancyfoot[R]{\setmainfont{Palace Script MT}\huge Marco Casu \setmainfont{Times New Roman}}
%==================FOOTER e HEADER=======================%

%==================INIZIO======================%
\chapter{Introduction}
In this chapter we will introduce the basics of what is a machine learning problem, giving a mathematical definition. informally, with machine learning we define the use of knowledge (data) to improve the performance of a given program, using past experiences.\bigskip

Generally, we use machine learning to solve problems with no deterministic solutions, trying to find an approximate one (such as recognizing what animal is represented in a given photo).\bigskip 

Usually, a machine learning problem consists in three main component:\begin{itemize}
    \item $T$ : the given task
    \item $P$ : a performance metric
    \item $E$ : the past experiences (the data)
\end{itemize}
Let's see an example, we consider the game \textit{Checkers}, and we want to model an agent that can play checkers.
\end{document}